\documentclass[12pt]{article}

% Paquetes esenciales
\usepackage[utf8]{inputenc}
\usepackage[spanish]{babel}
\usepackage{amsmath, amssymb, amsthm}
\usepackage{geometry}
\usepackage{fancyhdr}
\usepackage{titlesec}
\usepackage{setspace}
\usepackage{lastpage}
\usepackage{datetime2}

% Configuración de márgenes
\geometry{a4paper, left=2.5cm, right=2.5cm, top=2.5cm, bottom=2.5cm}

% Formato de secciones
\titleformat{\section}{\large\bfseries}{\thesection.}{1em}{}
\titleformat{\subsection}{\normalsize\bfseries}{\thesubsection.}{1em}{}

% Encabezado y pie de página
\pagestyle{fancy}
\fancyhf{}
\rhead{Tarea de Geometría}
\lhead{Universidad de Antioquia}
\rfoot{Página \thepage\ de \pageref{LastPage}}
\cfoot{\small \today}

% Interlineado
\onehalfspacing

% Información del autor
\newcommand{\nombre}{Daniel Soto}
\newcommand{\universidad}{Universidad de Antioquia}
\newcommand{\asignatura}{Geometría}
\newcommand{\fechahoy}{\today}

\begin{document}

% Portada (opcional, integrada en la primera página)
\begin{center}
    {\LARGE \textbf{\asignatura}} \\[0.5em]
        {\large \universidad} \\[0.3em]
            {\large Facultad de Ciencias Exactas y Naturales} \\[1em]
                {\large \nombre} \\[0.3em]
                    {\small \fechahoy}
                    \end{center}

                    \vspace{1.5em}

                    % Comienza el contenido de la tarea
                    \section*{Problema 1}
                    Escribe aquí el enunciado y tu solución. Puedes usar ecuaciones como:
                    \[
                    \int_a^b f(x)\,dx = F(b) - F(a)
                    \]

                    \section*{Problema 2}
                    Continúa con los demás ejercicios...

                    % Si deseas numerar los problemas, usa \section en lugar de \section*

                    \end{document}
