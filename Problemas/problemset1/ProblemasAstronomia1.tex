\documentclass[a4paper]{article}

\usepackage[spanish]{babel}
\usepackage{graphicx}
\usepackage{amsmath, amssymb}
\usepackage[margin=2cm]{geometry}
\usepackage{fancyhdr}
\usepackage{enumerate}
\usepackage[shortlabels]{enumitem}
\usepackage{parskip}
\usepackage[most]{tcolorbox}
\usepackage[hidelinks]{hyperref}
\usepackage{float}
\usepackage{array}

% cabecera
\pagestyle{fancy}
\fancyhead[l]{Daniel Soto}
\fancyhead[c]{Problemas de Astronomía \#1}
\fancyhead[r]{\today}
\fancyfoot[c]{\thepage}
\renewcommand{\headrulewidth}{0.2pt} % linea horizontal

\newcounter{problema}

\newtcolorbox[use counter=problema]{problema}[1][]{%
    colback=gray!15,
    colframe=gray!60,
    fonttitle=\bfseries,
    title={Problema~\theproblema:},
    boxrule=0.5mm,
    arc=3mm,
    boxsep=5pt,
    left=8pt,
    right=8pt,
    top=6pt,
    bottom=6pt,
    #1
}


\begin{document}

\begin{problema}
    
    \begin{figure}[H]
        \centering
        \includegraphics[width=0.7\textwidth]{TidalForce.png}
        \caption{Fuerza de marea entre la tierra y la luna}
        \label{fig:1}
    \end{figure}

    Una fuerza de marea es una diferencia en la intensidad de la gravedad entre dos puntos.
    El campo gravitacional de la luna produce una fuerza de marea a lo largo del diámetro de la Tierra, 
    lo que causa que la Tierra se deforme. También generan mareas de varios metros en la Tierra sólida, 
    y mareas aún más grandes en los océanos líquidos. La Figura \ref{fig:1} muestra una idea de la situación.

    Un ser humano cayendo en un agujero negro también experimentará fuerzas de marea. 
    ¡En la mayoría de los casos estas serán letales! La diferencia en la fuerza gravitacional 
    entre la cabeza y los pies podría ser tan intensa que una persona literalmente sería separada por tracción. 
    Algunos físicos han llamado a este proceso ¡\textit{espaguetificación}! \cite{algebra2}

    
    \begin{equation}
    a = \frac{2GMd}{R^3}
    \label{eq:1}
    \end{equation}

    \begin{enumerate}
        \item La ecuación \ref{eq:1} nos permite calcular la aceleración de marea $a$, através de un
        cuerpo de longitud $d$. La aceleración de marea entre tu cabeza y tus pies está dada por la fórmula \ref{eq:1}. 
        Para $M = 5.9\times 10^{27} \text{ g}$ (masa de la tierra), $R=6.4\times 10^8 \text{cm}$ (radio de la tierra) y
        $G=6.67\times 10^{-8} \text{ dyn} \cdot \text{cm}^2/\text{g}^2$, calcula la aceleración de marea, $a$, 
        si una altura humana típica es $d=200\text{ cm}$.
        \item ¿Cuál es la aceleración de marea a través del diámetro completo de la Tierra?
        \item Un agujero negro de masa estelar tiene la masa del sol $(1.9\times 10^{33}\text{ g)}$, 
        y un radio de $2.9 \text{km}$.
        \begin{enumerate}
            \item A una distancia de $100km$, ¿cuál sería la aceleración de marea através de un humano para $d=200cm$
            \item Si la aceleraciòn de gravedad en la superficie de la tierra es $980 	\text{cm/s^2}$, 
            ¿sería espaguetificado el desafortunado viajero humano cerca de un agujero negro de masa estelar?
        \end{enumerate}
        \item Un agujero negro supermasivo tiene $100$ millones de veces la masa del sol, y un radio de $295$ millones de kilómetros.
        ¿Cuál sería la aceleración de marea a través de un humano con $d=2\text{m}$, a una distancia de $100\text{km}$ desde el horizonte 
        de eventos del agujero negro supermasivo?
        \item En qué agujero negro podría entrar un humano sin ser espaguetificado?
    \end{enumerate}
    
\end{problema} [1.2.2]


\begin{problema}
    Aunque solo existen una docena de constantes físicas fundamentales de la naturalez,
    estas se pueden combinar para definir muchas otras constantes básicas en física, 
    química y astronomía.

    \begin{table}[H]
    \centering
    \begin{tabular}{|c|c|c|}
    \hline
    Símbolo & Nombre & Valor \\
    \hline
    $c$ & Velocidad de la Luz & $2.9979\times 10^{10} \text{ cm/s}$\\
    \hline
    $h$ & Constante de Planck & $6.6262\times 10^{-27} \text{ erg}\cdot \text{s}$\\
    \hline
    $m$ & Masa del Electrón & $9.1095\times 10^{-28} \text{ g}$\\
    \hline
    $e$ & Carga del Electrón & $4.80325 \times 10^{-10} \text{esu}$\\
    \hline
    $G$ & Constante de Gravitación & $6.6732 \times 10^{-8} \text{dyn} \cdot \text{cm}^2 \text{gm}^{-2}$\\
    \hline
    $M$ & Masa del Protón & $1.6726 \times 10^{-24} \text{g}$\\
    \hline
    \end{tabular}
    \caption{Constantes físicas}
    \label{tab:1}
    \end{table}
    En este ejercicio, calcularás algunas de estas constantes \textit{secundarias} con una precisión 
    de tres cifras significativas utilizando una calculadora o programación y los valores definidos 
    en la tabla \ref{tab:1}.

    \begin{enumerate}
        \item Constante de entropía de agujero negro $$\frac{c^3}{2hG}$$
        \item Constante de radiación gravitacional $$\frac{32 G^5}{5 c^{10}}$$
        \item Constante de Thomas-Fermi $$\frac{324}{175} \left(\frac{4}{9\pi}\right)^{2/3}$$
        \item Sección transversal de dispersión de Thompson $$\frac{8 \pi}{3} \left(\frac{e^2}{mc^2}\right)^2$$
        \item Límite de Stark $$\frac{1}{M^5} \left(\frac{4 \pi^2 e^2 m}{h^2}\right)^2$$
        \item Constante de radiación de Bremstrahlung $$\frac{32 \pi^2 e^6}{3\sqrt{2\pi}m^3c}$$
        \item Constante de Fotoionización $$\frac{32 \pi^2 e^6 (2\pi^2 e^4 m)}{3^{3/2} h^3}$$
    \end{enumerate}
\end{problema} [1.2.4]

\begin{problema}
    El $19$ de Julio de $1969$, el Módulo de Servicio y Comando Apollo-11 y el Módulo Lunar Eagle entraron en órbita lunar.
    \begin{figure}[H]
    \centering
    \includegraphics[width=0.5\textwidth]{moon.png}
    \label{fig:2}
    \end{figure}
    El tiempo necesario para completar una vuelta completa en la órbita se llama período orbital, 
    que en este caso fue de $2$ horas, a una distancia de $1.737$ km desde el centro de la Luna.

    ¡Crea o no lo creas, puedes usar estas dos piezas de información para determinar la masa de la Luna!
    Así es como se hace:

    \begin{enumerate}
        \item Suponga que el Apollo-11 entró en una órbita circular, y que la aceleración gravitacional hacia
        adentro ejercida por la Luna sobre la cápsula, $F_g$, equilibra exactamente la aeleración centrífuga
        hacia afuera, $F_c$. Resuelva $F_c=F_g$ para encontrar la masa de la Luna, $M$, en términos de $V$, $R$
        y la constante gravitacional $G$, dado que:
        $$F_g = \frac{GMm}{R^2}\quad\quad F_c = \frac{mV^2}{R}$$
        \item Usando el hecho que para el movimiento circular, $$V = \frac{2\pi R}{T}$$ reexprese su respuesta
        al problema 1 en términos de $R,T$ y $M$.
        \item Dado que $G=6.67\times 10^{-11} \text{ m}^3 \text{kg}^{-1} \text{s}^{-2}$, $R=1.737 \text{ km}$ 
        y $T=2\text{ h}$, calcule la masa de la Luna, $M$, en kilogramos.
        \item La masa de la Tierra es $M = 5.97\times 10^{24} \text{kg}$. ¿Cuál es la razón de la masa de la luna,
        derivada del problema 3, respecto a la pasa de la tierra?
    \end{enumerate}
\end{problema}[1.4.1]
\bibliographystyle{plainurl}
\bibliography{bibliografia} % Nombre del .bib

\end{document}
